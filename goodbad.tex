\chapter{The Good and Bad Multi-Sets}\label{chapter:goodbadset}

In this chapter we discuss certain subsets of multi-sets which will be important for us in future (to be specific in 
Section \ref{section:steptwo}). These sets will drive the entire reconstruction algorithm. Without too much delay we give the
definitions.

\begin{definition}
 Let $\MR,\MS,\MT$ be finite multi-sets in $\PP(V)$. For every non-negative integer $i$ we define $GoodSet^i_\MR(\MS,\MT)$ a sub multi-set of $\MS$
 as follows
 \[
  GoodSet^i_\MR(\MS,\MT) = \{s\in \MS : s\notin sp(\MR), |(sp(\MR\cup {s})\setminus sp(\MR)) \cap \MT| \leq i \}
 \]

\end{definition}
Let's explain this definition. Consider any $s\in \MS$, we add $s$ to the above set if and only if
\begin{itemize}
  \renewcommand\labelitemi{--}

\item $s$ does not belong to the flat $sp(\MR)$, and
 \item the flat generated by $s$ with $\MR$ i.e. $sp(\MR\cup \{s\})$ contains at most $i$ \textbf{distinct} points from $\MT$ outside $sp(R)$.
\end{itemize}
This can also be seen as generalizations of ordinary flats discussed in \cite{BDWY11}. We do not give too much explanation right now but 
give a picture for intuition. The bad sets are precisely the complements of good sets but we give a definition as it
comes very handy.

\begin{definition}
 Let $\MR,\MS,\MT$ be finite multi-sets in $\PP(V)$. For every non-negative integer $i$ we define $BadSet^i_\MR(\MS,\MT)$ a sub multi-set 
 of $\MS$ as follows
 \[
  BadSet^i_\MR(\MS,\MT) = \{s\in \MS : s\in sp(\MR) \text{ or } |(sp(\MR\cup {s})\setminus sp(\MR)) \cap \MT| > i \}
 \]

\end{definition}

\begin{remark}\label{remark:ordinary}
 Let's show how the above definition generalizes the notion of "ordinary flats". In \cite{BDWY11}, for a fixed finite set $V$,
 an ordinary $k$-flat was defined as a $k$-flat whose intersection with $V$ is contained in the union of a $(k-1)$-flat and
 a single point. Fix the finite set $V$ and let $\MR = \{v_1,\ldots,v_k\}$ be a linearly independent subset of size $k$. 
 Consider the subset $GoodSet^1_{\MR}(V,V)$, it contains all points $v_{k+1}$ in $V$ such that $v_{k+1}\notin sp\{v_1,\ldots,v_k\}$
 and $sp\{v_1,\ldots,v_k,v_{k+1}\}\setminus sp\{v_1,\ldots, v_k\}$ intersects $V$ in atmost one point i.e. the same as saying
 that the intersection of the $k$-flat $sp\{v_1,\ldots,v_k,v_{k+1}\}$ with $V$ is contained in the union of the $(k-1)$-flat
 $sp\{v_1,\ldots,v_k\}$ and the point $\{v_{k+1}\}$. So $GoodSet^1_\MR(V,V)$ is the collection of all "ordinary flats" which contain the
 $(k-1)-$flat $sp\{v_1,\ldots,v_k\}$. We also want to comment that our definition can be seen as a coloured version of "ordinary flats".
\end{remark}



By definition $GoodSet^i_\MR(\MS,\MT)$ and $BadSet^i_{\MR}(\MS,\MT)$ are complements of each other. We will first show some elementary results 
about these sets.

\begin{lemma}[Elementary Lemma]\label{lemma:elementary}
 Let $\MS\subset \MS^\prime$ and $\MT\subset \MT^\prime$. The following are true:
 \begin{enumerate}
  \item $GoodSet^i_{\MR}(\MS,\MT) \subset GoodSet^i_{\MR}(\MS^\prime,\MT) \hspace{0.5em}\text{ and }\hspace{0.5em}
  BadSet^i_{\MR}(\MS,\MT) \subset BadSet^i_{\MR}(\MS^\prime,\MT)$.
  \item $GoodSet^i_{\MR}(\MS,\MT) \supset GoodSet^i_{\MR}(\MS,\MT^\prime) \Rightarrow 
  BadSet^i_{\MR}(\MS,\MT) \subset BadSet^i_{\MR}(\MS,\MT^\prime)$.
 \end{enumerate}
\end{lemma}
\begin{proof}
 The proofs are straight-forward and we leave them as an exercise.
\end{proof}
The next lemma tells us that these good and bad multi-sets can be efficiently computed if the multi-sets
$\MR,\MS,\MT$ are known.
\begin{lemma}
 Suppose we know the multi-sets $\MR,\MS,\MT$. For every $i$, we can compute the sets $GoodSet^i_\MR(\MS,\MT)$
 and $BadSet^i_\MR(\MS,\MT)$ in time $poly(|\MR|, |\MS|, |\MT|,n)$ (note here $n$ is the dimension of the space $V$).
\end{lemma}
\begin{proof}
 This is pretty straight-forward. We iterate through all $s\in \MS$. We check if $s\in sp(R)$\footnote{ This can be done
  by solving a system of equations in $\leq |\MR|$ variables. }. If it is, then we go to the next $s$. Else we iterate through
  $t\in \MT$ and count the number of distinct $t$'s which belong to $sp(R \cup \{s\})\setminus sp(R)$.\footnote{ This can also
  be done by solving a system of equations in $\leq |\MR|+1$ variables. }. Finally if this number is $\leq i$ we add $s$ to the multi-set otherwise
  we go to the next $s$. This clearly takes $poly(|\MR|,|\MS|,|\MT|,n)$ time. Bad Set is just the complement of Good Set 
  inside $\MS$ and we know $\MS$ so this is also do able in $poly(|\MR|,|\MS|,|\MT|,n)$ time.
\end{proof}


\subsection{Coming back to $\Sigma\Pi\Sigma(2)$ circuits}

Let's come back to $\Sigma\Pi\Sigma(2)$ circuits. Recall the definition of the three multi-sets $\PP(R),\PP(B),\PP(Gcd(C))$ inside the
projective space of linear forms $\PP(V)$ given in Definition \ref{definition:definitionsandobservations}. Also from the same definition
recall the factorization 
\[C(\B{x}) = Gcd(C)Int(C)Res(C),\]
where $Gcd(C)Int(C)$ were product of all the linear factors of $C(\B{x})$ and
$Res(C)$ was the product of all the non-linear irreducible factors of $C(\B{x})$. Thus we may also define the multi-set 
$\PP(Gcd(C)Int(C))$ containing all projective linear factors of $C(\B{x})$. In order to prove certain important lemmas we will
also need the set of \emph{factoring forms} we defined in Definition \ref{definition:factoringform} along with Theorem 
\ref{theorem:candidatestructure} and Lemma \ref{lemma:matchinglemma} from Chapter \ref{chapter:sps2definitions}, 
about the structure of the set of factoring forms for
$Res(C)$ i.e. $\MP(Res(C))$. For shorthand let's denote $\MP= \MP(Res(C))$. Now we are ready to give the first interesting
lemma.

From here onwards we fix $\MR = \{r_1,\ldots,r_k\} \subset \PP(R)$ to be a linearly independent subset of 
size $k = R(3,d,\F)+2$. This inherently
assumes that $srank(C)\geq R(3,d,\F)+2$ and so we can use Theorem \ref{theorem:candidatestructure} and Lemma \ref{lemma:matchinglemma}
from Chapter \ref{chapter:sps2definitions}.
To make the expressions look simpler we define $\PP(R)^g = GoodSet^1_\MR(\PP(R),\PP(R))$ and $\PP(R)^b = BadSet^1_\MR(\PP(R),\PP(R))$.
These sets have gemetrical significant as explained in Remark \ref{remark:ordinary} above.

\begin{framed}
\begin{note}
We can reverse the roles of $R$ and $B$ in this entire subsection i.e. all lemmas continue to hold with 
 the roles reversed. In our application we might need it for any of them. For better exposition we do not repeat the arguments.
\end{note}
 \end{framed}



At first these lemmas might look rather confusing. But the proof will give a geometric intuition as to why they are true.
\begin{lemma}\label{lemma:factors}
The following hold:
\begin{equation}\label{equation:onlygcdfactors}
  \PP(Int(C)) = BadSet^1_{\MR}(\PP(Int(C)),\MP)) \subset BadSet^1_{\MR}(\PP(Gcd(C)Int(C)),\MP) 
\end{equation}
\begin{equation}\label{equation:projectionsize}
 BadSet^1_{\MR}(\PP(Gcd(C)Int(C)),\MP)\subset BadSet^0_{\MR}(\PP(Gcd(C)Int(C)),\PP(R)^b\cup\PP(B)) 
\end{equation}
 
In the above relation \ref{equation:projectionsize}, by $supp(\MX)$ means support of the multi-set $\MX$ i.e.
the set of distinct members of $\MX$.
\end{lemma}

\begin{proof}
\begin{enumerate}
 \item For the first equality, the direction $BadSet^1_{\MR}(\PP(Int(C),\MP))\subset \PP(Int(C))$ 
 is obvious by definition. For the other direction, we pick $p\in \PP(Int(C))$. If $p\in sp\{r_1,\ldots,r_k\}$ then
 $p\in Badset^1_{\MR}(\PP(Int(C)), \MP)$ and we are done. Else consider the set $sp\{r_1,\ldots,r_k,p\}\setminus sp\{r_1,\ldots,r_k\}$.
 We show that it contains $\geq 2$ forms from $\MP$. This will complete the proof. Since $p\in \PP(Int(C))$, it divides the
 polynomial $Sim(C)=R+B$ (See Definition \ref{definition:definitionsandobservations}) and thus on restriction to $ker(p)$ we get
 \[
  R_{|_{ker(p)}} + B_{|_{ker(p)}} = 0 \Rightarrow R_{|_{ker(p)}} =- B_{|_{ker(p)}}
 \]
Both $R_{|_{ker(p)}}$ and $B_{|_{ker(p)}}$ are non-zero otherwise $p$ becomes a common factor (recall $gcd(R,B)=1$). 
We know that $k\geq 3$ and thus $r_1,r_2,r_3$ are three distinct linearly independent points in $\MR$. $r_1,r_2,r_3$ 
 divide $R$ and so there must exist $b_1,b_2,b_3$ such that ${r_i}_{|_{ker(p)}} \sim {b_i}_{|_{ker(p)}}, i\in [3]$. Thus
$r_i,b_i,p$ are collinear for $i\in [3]$. If the set $\{b_1,b_2,b_3\}$ had dimension $1$ then $\{r_1,r_2,r_3\}\subset sp\{b_1,b_2,b_3,p\}$
has dimension $\leq 2$ which is a contradiction. Thus atleast two of the points say $b_1,b_2$ are distinct. The forms $r_i,b_i,p$
are collinear and distinct implying that $b_i \in sp\{r_i,p\}\subset sp\{r_1,\ldots,r_k,p\}, i\in [2]$. If $b_i\in sp\{r_1,\ldots,r_k\}$
by the same argument $p\in sp\{r_i,b_i\}\subset sp\{r_1,\ldots,r_k\}$ which is a contradiction to the choice of $p$. Therefore
$b_1,b_2$ are two distinct points in $(sp(\MR\cup\{p\})\setminus sp(R))\cap \MP$. (Note: $b_1,b_2$ belong to $\MP$ by Theorem 
\ref{theorem:candidatestructure} which states that distinct forms from $\PP(B)$ are in $\MP$).

The second containment follows by using Lemma \ref{lemma:elementary} given earlier in this chapter since
$\PP(Int(C))\subset \PP(Gcd(C)Int(C))$.
 
 
 \item For this containment let $p\in BadSet^1_{\MR}(\PP(Gcd(C)Int(C)),\MP)$. If $p\in sp(\MR)$ then it belongs to
 the RHS by definition. Else consider the set $sp(\MR\cup\{p\})\setminus sp(\MR)$. We know there are atleast two distinct points
 from $\MP$ on this set. Call them $p_1,p_2$. We will show that there is at least one point from $\PP(R)^b \cup \PP(B)$ on this set.
 If any of $p_1,p_2$ belongs to $\PP(B)$ we are done. If any of $p_1,p_2$ say $p_1$ is a point outside $\PP(B)\cup \PP(R)$
 by Matching Lemma (See Lemma \ref{lemma:matchinglemma} in Chapter \ref{chapter:sps2definitions}) we know that for some $i\in [k]$
 there is a $b_i$ in $\PP(B)$ such that $r_i,b_i,p_1$ are collinear. Exactly like the argument above we can show that
 $b_1 \in sp\{r_1,\ldots,r_k,p_1\}$ and $b_1\notin sp\{r_1,\ldots,r_k\}$ (since $p_1$ was outside this flat). Therefore
 we found a $b_1 \in sp(\MR\cup \{p\})\setminus sp(\MR) \Rightarrow$ and we are done. So we are left with the case when both 
 $p_1,p_2$ are in $\PP(R)$. Since there are two distinct points $p_1,p_2\in \PP(R)$ on the set $sp(\MR\cup\{p\})\setminus sp(\MR)$
 we can say that $p_2\in sp(\MR\cup\{p_1\})\setminus sp(\MR)$ and thus $p_2\in Badset^0_{\MR}(\PP(R),\PP(R)) = \PP(R)^b$. So
 there is a point from $\PP(R)^b$ on $sp(\MR\cup \{p_1\})\setminus sp(\MR)$ and we are done.
\end{enumerate}

 

\end{proof}




We prove another lemma of the same flavor below. To make the expressions look simpler we define 
$\MG = BadSet^1_{\MR}(\PP(Gcd(C)Int(C)),\MP)$, the set involved in the previous lemma.
\begin{lemma}\label{lemma:overestimatedetector}
 The following hold:
 \begin{equation}\label{equation:largesizeredpoints}
  GoodSet^0_\MR(\PP(R)^g, \PP(B)) = GoodSet^0_\MR(\PP(R)^g,\PP(B)\cup \MG)
 \end{equation}
 \begin{equation}\label{equation:overestimate}
  GoodSet^0_\MR(\MP,\PP(B)\cup \MG) \subset \PP(R)
 \end{equation}
\end{lemma}

\begin{proof}
 \begin{enumerate}
  \item As $\PP(B) \subset \PP(B)\cup\MG$, by Lemma \ref{lemma:elementary} one direction is obvious i.e.
  $GoodSet^0_\MR(\PP(R)^g, \PP(B)) \supset GoodSet^0_\MR(\PP(R)^g,\PP(B)\cup \MG)$. For the other direction,
  consider $p\in GoodSet^0_\MR(\PP(R)^g, \PP(B))$. Clearly $p\notin sp(\MR)$ since it is in the good set. Also
  $p\in \PP(R)^g$ by definition. We know that $sp(\MR\cup \{p\})\setminus sp(\MR)$ does not intersect $\PP(B)$ by the
  choice of $p$. If we show that it also does not intersect $\MG$ we will be done since that implies the containment.
  Assume it does intersect $\MG$ i.e. there exists $g_1\in \MG$ belonging to $sp(\MR\cup\{p\})\setminus sp(\MR)$. 
  Since $\MG = BadSet^1_{\MR}(\PP(Gcd(C)Int(C)), \MP)$ we know there exist two distinct $p_1,p_2$ from $\MP$ on
  $sp(\MR\cup\{g_1\})\setminus sp(\MR) = sp(\MR\cup\{p\})\setminus sp(\MR)$. Now if any of the $p_1,p_2$ are in $\PP(B)$
  we get a contradiction since we just showed above that $sp(\MR\cup\{p\})\setminus sp(\MR)$ does not intersect $\PP(B)$. 
  If any of $p_1,p_2$ say $p_1$ is outside $\PP(R)\cup \PP(B)$ exactly like the proof of the previous lemma, using the Matching
  Lemma (See Lemma \ref{lemma:matchinglemma}) we conclude that there is a $b_1\in \PP(B)$ on the set $sp(\MR\cup \{p\})\setminus
  sp(\MR)$ (we do not repeat the argument here, please see the second part of previous lemma) but that is a contradiction
  to this set not intersecting $\PP(B)$. Therefore $p_1,p_2$ are distinct points from $\PP(R)$. But $p\in \PP(R)^g$ i.e. 
  the set $sp(\MR\cup\{p\})\setminus sp(\MR)$ can contain at most one from from $\PP(R)$ and therefore we have a contradiction.
  So $sp(\MR\cup\{p\})\setminus sp(\MR)$ does not intersect $\MG$ 
  implying that $GoodSet^0_\MR(\PP(R)^g, \PP(B)) \subset GoodSet^0_\MR(\PP(R)^g,\PP(B)\cup \MG)$ and we are done.
  
  \item Let $p\in GoodSet^0_\MR(\MP,\PP(B)\cup \MG)$. Clearly this implies that $p\notin sp(\MR)$. By the choice of $p$
  it belongs to $\MP$. It cannot be in $\PP(B)$ since that would imply that $sp(\MR\cup\{p\})\setminus sp(\MR)$
  intersects $\PP(B)$ contradicting the choice of $p$. If $p$ outside $\PP(R)\cup \PP(B)$ exactly like the second parts
  of the previous two lemmas we use the Matching Lemma (See Lemma \ref{lemma:matchinglemma}) to conclude that there is a $b_1\in \PP(B)$
  on the set $sp(\MR\cup\{p\})\setminus sp(\MR)$ giving a contradiction to the choice of $p$ like before. Therefore $p\in \PP(R)$
  and we are done.
  
 \end{enumerate}

\end{proof}




We also include a simple corollary to the above Lemma \ref{lemma:overestimatedetector}.

\begin{corollary}
Using Lemma \ref{lemma:overestimatedetector} and Lemma \ref{lemma:elementary} we can conclude that
\begin{equation}
 GoodSet^0_{\MR}(\PP(R)^g, \PP(B)) \subset GoodSet^0_{\MR}(\MP,\PP(B)\cup\PP(G))\subset \PP(R) 
 \end{equation}

\end{corollary}

\begin{proof}
 The argument here is straight-forward. We know that $GoodSet^0_{\MR}(\PP(R)^g, \PP(B)) = 
 GoodSet^0_{\MR}(\PP(R)^g, \PP(B)\cup \MG)$ by the lemma above. $\PP(R)^g\subset \PP(R)\subset \MP$
 and therefore Lemma \ref{lemma:elementary} implies that $GoodSet^0_{\MR}(\PP(R)^g, \PP(B)\cup \MG)\subset
 GoodSet^0_{\MR}(\MP, \PP(B)\cup \MG)$ and we are done.   
\end{proof}



Finally we describe an algorithm to compute $GoodSet^0_{\MR}(\MP,\PP(B)\cup \MG)$ involved in Lemma \ref{lemma:overestimatedetector}
above. For a certain choice of $\MR$ this set will play a very important role in reconstruction.
This algorithm assumes that we have access to multi-sets $\MR, \MP$ and black-box access to the polynomial $C(\B{x})$.





